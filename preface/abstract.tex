\cabstract{

    对于每一名编程学习者而言,编程练习是掌握相关知识的必经之路。编程学习者进行编程练习的目的各不相同,也许是为了提高编程的技术水平,也许是为了准备企业的笔试和面试,又也许是为了完成一个科目的大作业……

    为了满足编程练习的需求,出现了在线评测网站,这些网站提供了一些编程题,有相关学习需求的人可以在线提交编程题的答案,并获得网站评测后的结果。但是,目前市面上所有的在线评测系统都具有一些共性的问题。例如,它们都采用集中评测的方式,当大流量的评测事件发生时,用户并不能在期望的时间内完成评测。另外,已有的评测系统太过纯粹了,学习者只能在评测网站上验证已知的知识,新知识的获取则需另寻他途。

    为了解决这些问题,我们设计并开发了主要基于Electron的新一代编程学习平台——PhoeniX。PhoeniX提供了编程学习的一体化解决方案,PhoeniX包括教程共享、在线评测、组织管理、比赛等子系统。相比于市面上的在线评测系统,PhoeniX的题目评测在用户端分布式进行,用户仅需将评测结果上传至云端,避免了在服务端进行评测时的卡顿现象,可以良好地应对各种大型比赛活动的评测需求。此外,PhoeniX提供了知识共享服务,用户可以在讨论区分享对于题目的见解,也可以在教程区发布编程学习的交互式的教程,以供其他用户交流学习。
}

\ckeywords{分布式,在线评测,编程学习,知识共享}

\eabstract{
    For every programming learner, programming practice is the only way to master relevant knowledge.  Programming learners have different purposes for programming practice, perhaps to improve their programming skills, perhaps to prepare for the written test and interview of the enterprise, or perhaps to finish a large assignment for a subject...  

    For meeting the needs of programming practice, online evaluation websites have emerged. These websites provide some programming questions, and people with pertinent learning needs can submit the answers to the programming questions online, and get the results after the website evaluation.  However, all online evaluation systems on the market have some common problems.  For example, they all adopt a centralized profiling approach, and when a high-volume profiling event occurs, users are not able to complete the profiling in the expected time.  In addition, the existing evaluation system is too pure, learners can only verify the known knowledge on the evaluation website, and the acquisition of new knowledge has to find another way.  
    
    In order to solve these problems, we design and develop a new generation platform of programming learning, PhoeniX, which is mainly based on Electron.  PhoeniX provides an integrated solution for programming learning, including tutorial sharing, online evaluation, organization management, competition, and other subsystems.  Compared with online evaluation systems in the market, PhoeniX subject evaluation is distributed on the client-side, so users only need to upload the evaluation results to the cloud, avoiding the lag phenomenon during evaluation on the server-side, and can well meet the evaluation requirements of various large-scale competitions.  Besides, PhoeniX provides a knowledge-sharing service where users can share their insights on topics in the discussion area and publish interactive tutorials for programming learning in the tutorial area for other users to exchange and learn.  
}

\ekeywords{Distributed, Online Judge, Programming Learning, Knowledge Sharing}

\makecover

\clearpage


\chapter{部分实现细节}

\section{评测数据的处理}

用户编写的题目会带有评测数据,即题目的输入和输出。分布式评测需要将输入和输出(或者数据生成器和标准程序)传输到用户机上,再由用户机进行题目的评测。在大部分情况下输入和输出对于用户的可见性并不重要,但是如果数据不对用户保密,就可能会有用户得到题目的输入和输出后直接根据测试点编程的情况。

对于上述这种情况,我们在客户端和服务端都采取了措施。当用户上传题目时,我们会对题目的数据进行加密,加密的算法和密钥由服务端指定。当用户下载题目时,我们将提供给客户端解密的密钥和解密算法的必要信息。当用户评测完题目时,我们会将用户本地的评测数据删除,以此确保数据的相对安全性。

另外,如果用户通过某种途径(如HTTPS抓包)获取了服务端的评测接口,想通过直接调用接口的形式仿造题目评测,这也是有可能的。我们在服务端对这种情况也进行了处理,即用户在评测题目时会将用户的程序上传到服务端,服务端可以再次对这些代码进行评测,验证这些程序的合法性。对于组织创建的比赛,我们提供了对用户上传的程序进行重测的接口,保证比赛系统的公正性。

如果人力等资源允许,事实上还可以通过训练人工智能模型,来进一步判定用户代码是否是根据题目输出编程得来的,或者是伪造的。考虑到根据题目输出编程得来的程序的特征(源代码中包含大量标准输出),使用正则表达式匹配就能筛选出大部分这类程序。这一部分的工作我们仍处于筹划阶段,具体实现方法仍待考虑。

\section{教程的渲染}

PhoeniX给用户呈现的教程需要能够嵌入代码执行环境,渲染Latex公式以及Markdown语法,保证渲染后的教程对用户而言没有安全风险。经过考量之后,我们选用了以markdown-it渲染器为核心的渲染方案,原因有如下几点:

\begin{enumerate}
    \item 教程的渲染需要较高的效率,而且之后可能需要支持更多的语法与渲染内容,引入第三方封装的markdown编辑器做渲染是明显不合适的,这些编辑器的渲染效率不高,且难以自定义渲染效果。
    \item markdown-it渲染器本身足够轻量,使用者和维护者都较多,能够识别并正确渲染HTML保证用户安全,且支持自定义插件,容易自定义渲染效果。
    \item markdown-it有一些现成的足够轻量的插件,分别能够支持Latex渲染、Markdown扩展语法渲染等任务,不需要重新手写太多插件。
\end{enumerate}

然而,markdown-it渲染器只能完成最基本的渲染任务,对于内嵌的代码执行环境,我们是使用Monaco编辑器和Xterm虚拟终端实现的。Monaco编辑器和Xterm终端是微软的开源项目Visual Studio Code(以下简称VSCode)解决方案的一部分,考虑到VSCode给用户提供的绝佳体验,我们认为VSCode的编辑器解决方案是成功的。在PhoeniX的具体实现上,我们对Monaco和Xterm都进行了一定程度的自定义,使其能够较为美观的呈现,并保有执行我们提供的命令的基本功能。

\section{推荐算法的设计}

我们设计了一套基于情景感知的题目推荐模型。 我们的设计理念是要尽可能的推荐用户最有可能查看并完成的题目。 即, 规定事件 $X$ 为用户对某个题的提交。 $Y$ 表示用户当前的 “情景”。 其构成包括: 用户近期所做过的题目、 题目标签、 题目难度、 用户所属的组织、 组织成员的做题偏好等。 我们的目标则是最大化 $P(X|Y)$。

具体而言,我们针对特定情境寻找题目时,会计算下列信息:

\begin{enumerate}
    \item 提取出用户近期提交过的题目标签。 提升拥有相同标签的题目的权重。
    \item 根据用户最近过题的难易程度,降低难度较高或较低的题目权重。
    \item 查看用户所属组织成员的做题情况, 提升组织内其他成员做过的题目的权重。
    \item 系统内预设一套学习路线, 根据学习路线预测用户接下来可能学习的知识点。 提升其模板题的权重。
\end{enumerate}

我们将题目按照权重排序,选择权重最高的题目返回给用户。 实现个性化的推荐功能。 此外, 我们还会全局统计每个题的近期过题量, 找出近期的热门题目, 以排行榜的形式展示。 这也可以作为用户选择题目的参考。

在具体的实现中, 由于不同影响因素的重要程度难以衡量, 我们会尽可能的给用户更多样化的选择, 具体而言。 当有很多权重很高的题目时, 我们会随机删除掉部分权重很高但是题目标签重复出现的题目。 来保证我们推荐的题目的多样性。
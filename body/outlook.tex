\chapter{前景展望}

\section{项目已取得的成果}

\begin{enumerate}
\item{一体化的编程学习平台}
\par
本项目PhoeniX是主要使用Electron框架搭建的桌面端应用,解决了在编程学习中各类资源与评测网站过于分散、用户等待题目评测的时间过长等问题,集编程教学、在线评测、轻社交于一体,为编程学习者创造了一个知识充分共享、评测体系完备、团队交流友好的学习环境。

\item{多语言分布式评测}
\par
相比传统的在线评测网站,PhoeniX为用户提供分布式评测的功能,并有效降低了中心服务器的评测压力。只要在远程配置不同题目的评测脚本,就可以让用户的程序在PhoeniX提供的本地环境下编译并运行,并在本地给出评测的结果。由于程序在本地而不是在远程编译运行,PhoeniX还支持更复杂的多文件、多语言评测的功能,让评测的项目种类有了更多可能。

\item{可交互式的教程}
\par
PhoeniX的教程设计哲学与Jupyter Notebook一致,即所有交互与计算、说明文档、数学公式、图片以及其他富媒体形式的输入和输出,都是以文档的形式体现的。教程编写者只需要撰写一份带有标记格式的文档,在PhoeniX中就可以被解析为一篇支持代码缩进高亮、编译执行的教程页面,供编程学习者自由学习和实验。

\item{组织管理与个人表达}
\par
在课堂教学和各类算法竞赛的实践中,参与者往往有在编程时组队、实时排名的需求。因此PhoeniX允许用户自由地创建组织,并在组织内分享比赛和教程。在教学实践中,一些教程和比赛仅对于一部分学生开放,这可以通过PhoeniX的组织权限管理功能轻松实现。除此之外,学生在学习中遇到的困难或是想要分享的技术都可以在公开的论坛或是组内进行讨论。
\end{enumerate}
\newpage

\section{项目仍存在的问题}

\begin{enumerate}
    \item 项目仍处于迭代开发阶段,每次迭代更新的新版本发布后需要用户重新下载。
    \item UI的设计和易用性上还有优化的空间。
    \item 使用Electron框架导致安装包较大,会考虑缩减一些不必要的依赖。
\end{enumerate}

\section{未来的目标}

对于未来版本的PhoeniX的升级计划,我们的目标如下:

\begin{enumerate}
    \item 引入智能推荐的系统。根据用户近期的做题情况,查看教程情况,量身定制推荐题库及教程。
    \item 搜索结果的智能排序。用户在搜索教程/问题时,按照教程/问题的权重进行排序,并返回给用户。
    \item 自动化的上传内容审查。用户上传教程/问题时,自动对内容进行审核,提高平台智能化程度。
\end{enumerate}

我们希望PhoeniX提供的编程学习解决方案能够真正地提高同学们在学习编程和算法时的体验。

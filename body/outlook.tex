\chapter{前景展望}

\section{项目已取得的成果}
\begin{enumerate}
    

\item{教学测一体化的平台}
\par
本项目Phoenix是使用Eletron框架搭建的桌面端应用,针对学生在学习校内课程各类资源或是评测网站过于分散的问题进行开发,集教程、评测、社交于一体,为学生搭建了一个方便使用、评测快捷、促进交流的学习环境。
\item{多语言分布式评测}
\par
PhoeniX因为其桌面端软件的优越性,相比传统的评测网站可以为用户提供分布式评测的功能,并有效降低中心服务器的评测压力。只要在远程配置不同题目的评测脚本,就让用户的评测程序在Phoneix提供的本地环境下编译并运行,并在本地给出评测的结果。因为程序在本地编译运行的方法,Phoenix还支持更复杂的多文件、多语言评测的功能,让评测的程序种类有了更多可能。

\item{可交互式的教程}
\par
Phoenix的教程设计哲学与JupyterBook一致,其所有交互计算、编写说明文档、数学公式、图片以及其他富媒体形式的输入和输出,都是以文档的形式体现的。教程编写者只需要撰写一份带有标记格式的文档,在Phoenix中就可以被解析为一篇支持代码缩进高亮、编译执行的教程页面,供教程学习者自由学习和实验。
\item{组织管理与个人表达}
\par
在课程教学和各类算法竞赛的实践中,学生往往有在编程时组队、实时排名的需求。因此Phoenix让用户可以自由创建组织,并在组织内分享比赛和教程。在教学实践中,对于一些教程和竞赛的开放权限也可以通过组织成员的权限管理轻松在班级中投送内容。除此之外,学生在学习中遇到的困难或是想要分享的技术都可以在公开的论坛或是组内进行讨论。
\end{enumerate}
\newpage

\section{项目仍存在的问题}
\begin{enumerate}
    \item 项目仍在更新开发状态,每次迭代更新后的新版本发布后需要用户重新下载。
    \item UI的排版和易用性上还有优化的空间。
    \item 使用Eletron框架导致安装包体较大,会考虑缩减一些不必要的依赖。
\end{enumerate}
\section{未来的目标}
对于未来版本的Phoenix的升级计划,我们的目标如下:
\begin{enumerate}
    \item 引入智能推荐的系统。根据用户近期的做题情况,查看教程情况,量身定制推荐题库及教程。
    \item 搜索结果的智能排序。用户在搜索教程/问题时,按照教程/问题的权重进行排序,并返回给用户。
    \item 自动化的上传内容审查。用户上传教程/问题时,自动对内容进行审核,提高平台智能化程度。
\end{enumerate}
我们希望Phoenix提供的教学测平台的一体化解决方案能够真正地优化同学们在学习编程和算法时的体验。
